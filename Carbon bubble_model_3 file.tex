%2multibyte Version: 5.50.0.2953 CodePage: 1251

\documentclass{article}
%%%%%%%%%%%%%%%%%%%%%%%%%%%%%%%%%%%%%%%%%%%%%%%%%%%%%%%%%%%%%%%%%%%%%%%%%%%%%%%%%%%%%%%%%%%%%%%%%%%%%%%%%%%%%%%%%%%%%%%%%%%%%%%%%%%%%%%%%%%%%%%%%%%%%%%%%%%%%%%%%%%%%%%%%%%%%%%%%%%%%%%%%%%%%%%%%%%%%%%%%%%%%%%%%%%%%%%%%%%%%%%%%%%%%%%%%%%%%%%%%%%%%%%%%%%%
\usepackage{amsmath}
\usepackage{amsfonts}

\setcounter{MaxMatrixCols}{10}
%TCIDATA{OutputFilter=LATEX.DLL}
%TCIDATA{Version=5.50.0.2953}
%TCIDATA{Codepage=1251}
%TCIDATA{<META NAME="SaveForMode" CONTENT="1">}
%TCIDATA{BibliographyScheme=Manual}
%TCIDATA{Created=Friday, August 02, 2019 09:55:10}
%TCIDATA{LastRevised=Wednesday, August 28, 2019 11:45:15}
%TCIDATA{<META NAME="GraphicsSave" CONTENT="32">}
%TCIDATA{<META NAME="DocumentShell" CONTENT="Standard LaTeX\Blank - Standard LaTeX Article">}
%TCIDATA{CSTFile=40 LaTeX article.cst}

\newtheorem{theorem}{Theorem}
\newtheorem{acknowledgement}[theorem]{Acknowledgement}
\newtheorem{algorithm}[theorem]{Algorithm}
\newtheorem{axiom}[theorem]{Axiom}
\newtheorem{case}[theorem]{Case}
\newtheorem{claim}[theorem]{Claim}
\newtheorem{conclusion}[theorem]{Conclusion}
\newtheorem{condition}[theorem]{Condition}
\newtheorem{conjecture}[theorem]{Conjecture}
\newtheorem{corollary}[theorem]{Corollary}
\newtheorem{criterion}[theorem]{Criterion}
\newtheorem{definition}[theorem]{Definition}
\newtheorem{example}[theorem]{Example}
\newtheorem{exercise}[theorem]{Exercise}
\newtheorem{lemma}[theorem]{Lemma}
\newtheorem{notation}[theorem]{Notation}
\newtheorem{problem}[theorem]{Problem}
\newtheorem{proposition}[theorem]{Proposition}
\newtheorem{remark}[theorem]{Remark}
\newtheorem{solution}[theorem]{Solution}
\newtheorem{summary}[theorem]{Summary}
\newenvironment{proof}[1][Proof]{\noindent\textbf{#1.} }{\ \rule{0.5em}{0.5em}}
\input{tcilatex}
\begin{document}


\begin{center}
{\LARGE Title: Carbon bubble...}

Last modified: 28 August 2019 by Fra
\end{center}

\section{The model}

There are nine types of agents in the model: households, financial
intermediaries, clean and dirty capital good producers, clean and dirty
intermediate goods producers, final good monopolistically competitive
producers, a central bank conducting monetary policy, and a government
setting fiscal and environmental policies.

\subsection{Households}

There is a continuum of representative households in the space $[0,1]$.
Within the household a fraction $1-f$ of members are workers and a fraction $%
f$ are bankers. Workers supply labor to firms in the two intermediate goods
producing sectors and return their wage to the household, bankers manage
financial intermediaries and, in turn, transfer earnings back to the
respective household.

With i.i.d probability $1-\U{3b8} $, a banker exits next period. Upon
exiting, a banker transfers retained earnings to the household and becomes a
worker. Each period, $(1-\U{3b8} )f$ workers randomly become bankers,
keeping the proportion of each occupation constant\footnote{%
This assumption rules out equilibria with full internal financing
(deposits=zero). A fraction of the earnings is retained by bankers. To avoid
a situation in which bankers can fund all investments from own accumulated
capital, and are thus not financially constrained anymore, we introduce a
turnover between bankers and workers.}. Each new banker receives a start up
transfer from the family.

Household derives utility from consumption, $C_{t},$ and disutility from
hours worked, $L_{t}$, where $L_{t}=L_{c,t}+L_{d,t}$. The lifetime utility $%
U $ is of the type:

\begin{equation}
U_{0}=\mathbb{E}_{0}\sum_{t=0}^{\infty }\beta ^{t}\left[ \ln \left(
C_{t}-h_{c}\overset{\_}{C}_{t-1}\right) -\chi _{l}\frac{L_{t}^{1+\varphi
_{l}}}{1+\varphi _{l}}\right] ,  \label{utility}
\end{equation}%
where $\beta \in (0,1)$ is the discount factor, $\chi _{l}>0$ is a scale
parameter measuring the relative disutility of labor, and $\varphi _{l}>0$
is the inverse of the Frisch elasticity of labor supply.

Preferences display external habit formation (i.e. \textquotedblleft
catching up with the Joneses\textquotedblright\ preferences (Abel 1990)),
where $h_{c}\in $ $[0,1)$ is the habit coefficient and $\overset{\_}{C}%
_{t-1} $is the lagged aggregate consumption of households (taken as given by
each household). Households pay lump-sum tax $T_{t}$ to the government, and
hold bank deposits (except deposits of the banks they own) and government
bonds. We assume that bank deposits and government bonds $B_{t}$ are one
period risk free bonds and are perfect substitutes, both paying a gross real
return $R_{t}$. Households receive net payouts $D_{t}$ from the ownership of
financial intermediaires, final-good producers and capital producing firms. $%
D_{t}$ corresponds also to the net amount the household gives to its members
entering the banking sector in period $t$.

The budget constraint of households reads as:

\begin{equation}
C_{t}=W_{t}(1-\tau _{W,t})L_{t}+D_{t}-T_{t}+R_{t-1}B_{t-1}-B_{t},
\label{h_budget}
\end{equation}

where $W_{t}$ is the real wage and $\tau _{W,t}$ is the labor-income tax
rate.

The problem of the typical household is

$\mathcal{L}_{0}^{H}=\mathbb{E}_{0}\sum_{t=0}^{\infty }\beta ^{t}\left\{ \ln
\left( C_{t}-h_{c}\overset{\_}{C}_{t-1}\right) -\chi _{l}\frac{%
L_{t}^{1+\varphi _{l}}}{1+\varphi _{l}}+\lambda _{t}\left[ W_{t}(1-\tau
_{W,t})L_{t}+D_{t}-T_{t}+R_{t-1}B_{t-1}-B_{t}-C_{t}\right] \right\} $

FOC wrt $C_{t}$

$\frac{1}{C_{t}-h_{c}\overset{\_}{C}_{t-1}}=\lambda _{t}$

FOC wrt $L_{t}$

$\chi _{l}L_{t}^{\varphi _{l}}=\lambda _{t}W_{t}(1-\tau _{W,t})$

FOC wrt $B_{t}$

$\beta \mathbb{E}_{t}\Lambda _{t,t+1}R_{t}=1$

with

$\Lambda _{t,t+1}=\frac{\lambda _{t+1}}{\lambda _{t}}$

Summing-up from the household's problem we keep the following equations:

\begin{itemize}
\item $C_{t}=W_{t}(1-\tau _{W,t})L_{t}+D_{t}-T_{t}+R_{t-1}B_{t-1}-B_{t}$
(household budget constraint)

\item $\frac{1}{C_{t}-h_{c}\overset{\_}{C}_{t-1}}=\lambda _{t}$ (marginal
utility of consumption)

\item $\chi _{l}L_{t}^{\varphi _{l}}=\lambda _{t}W_{t}(1-\tau _{W,t})$
(labor supply)

\item $\beta E_{t}\Lambda _{t,t+1}R_{t}=1$ \ (Euler's equation)

\item $\Lambda _{t,t+1}=\frac{\lambda _{t+1}}{\lambda _{t}}$ \ \ \ \
(discount factor)

\item $U_{t}=\ln \left( C_{t+i}-h_{c}\overset{\_}{C}_{t+i-1}\right) -\chi
_{l}\frac{L_{t+i}^{1+\varphi _{l}}}{1+\varphi _{l}}+\mathbb{E}_{t}\beta
U_{t+1}$ \ \ (welfare measure)
\end{itemize}

\bigskip

\subsection{Production side}

\subsubsection{Final good sector}

The final good sector is made up of a continuum of monopolistically
competitive  producers indexed by $j\in \lbrack 0,1]$.\textbf{(Ricordarsi di
inserire in nota letteratura che giustifica il fatto di avere conocrrenza
monopolistica nel settore finale)} The final good is produced by combining
clean ($c$) and dirty ($d$) inputs, $Y_{c}$ and $Y_{d}$, according to the
following technology:

\begin{equation}
Y_{t,j}=\left[ \rho ^{1/\varepsilon }\left( ef_{c}Y_{c,j,t}\right) ^{\left(
\varepsilon -1\right) /\varepsilon }+(1-\rho )^{1/\varepsilon }\left(
ef_{d}Y_{d,j,t}\right) ^{\left( \varepsilon -1\right) /\varepsilon }\right]
^{\varepsilon /(\varepsilon -1)},  \label{final_good}
\end{equation}%
where $\rho $ represents the share of clean intermediate goods used in the
production of final good, the coefficent $ef_{c}$, $ef_{d}$ measure input's
efficiency, and$\ \varepsilon >0$ is the elasticity of substitution between
the two intermediate inputs. When $\varepsilon >1$ the two inputs are
substitutes, when $\varepsilon <1$ are complements. In our benchmark
calibration we assume $\varepsilon >1$.

Final good produced varieties are combined into a final output composite $%
Y_{t}$, according to a CES function $Y_{t}=\left( \int_{0}^{1}\left(
Y_{j,t}\right) ^{\left( \sigma -1\right) /\sigma }dj\right) ^{\sigma
/(\sigma -1)}$, where $\sigma >1$ denotes the elasticity of substitution
between differentiated final goods. Cost minimization delivers the demand
schedule for variety $j$, $Y_{j,t}=\left( P_{j,t}/P_{t}\right) ^{-\sigma
}Y_{t}$, and the aggregate price index, $P_{t}=\left(
\int_{0}^{1}P_{j,t}{}^{(1-\sigma )}dj\right) ^{1/(1-\sigma )}$.

The objective of each firm $j$ is to maximize the sum of expected discounted
profits by setting the optimal price $P_{j,t}$ and making choices about
clean and dirty inputs, subject to the available technology, the price of
the intermediate inputs, $P_{c,t}$ and $P_{d,t}$, the demand schedule for
variety $j$, and quadratic adjustment costs on price setting:%
\begin{equation}
\Gamma _{p}(P_{j,t})=\frac{\gamma _{p}}{2}\left( \frac{1}{\Pi _{t-1}^{\kappa
_{p}}\Pi ^{1-\kappa _{p}}}\frac{P_{j,t}}{P_{j,t-1}}-1\right) ^{2}Y_{t},
\label{price_adjust}
\end{equation}%
where the coefficient $\gamma _{p}>0$ measures the degree of price rigidity, 
$\kappa _{p}\in \left[ 0,1\right] $ denotes the weight of past inflation in
the indexation, and $\Pi $ is the steady state inflation. We further assume
that final good producers may pay taxes or receive incentives in the form of
subsidies for using intermediate inputs.

Real profits of the $j$ final-good producer are then defined as:

$PRO_{j,t}=\frac{P_{j,t}}{P_{t}}Y_{j,t}-\frac{\gamma _{p}}{2}\left( \frac{1}{%
\Pi _{t-1}^{\kappa _{p}}\Pi ^{1-\kappa _{p}}}\frac{P_{j,t}}{P_{j,t-1}}%
-1\right) ^{2}Y_{t}-\frac{P_{c,t}}{P_{t}}\left( 1+\tau _{c,t}\right)
Y_{c,j,t}-\frac{P_{d,t}}{P_{t}}\left( 1+\tau _{d,t}\right) Y_{d,j,t}$

where $\tau _{c}$ and $\tau _{d}$ are the fiscal instruments (taxes if
positive, subsidies if negative).

The optmization problem is then:

$\mathcal{L}_{j,0}^{Firm}=\mathbb{E}_{0}\underset{t=0}{\overset{\infty }{%
\sum }}\beta ^{t}\Lambda _{0,t}\left\{ 
\begin{array}{c}
\frac{P_{j,t}}{P_{t}}Y_{j,t}-\frac{\gamma _{p}}{2}\left( \frac{1}{\Pi
_{t-1}^{\kappa _{p}}\Pi ^{1-\kappa _{p}}}\frac{P_{j,t}}{P_{j,t-1}}-1\right)
^{2}Y_{t}-\frac{P_{c,t}}{P_{t}}\left( 1+\tau _{c,t}\right) Y_{c,j,t}-\frac{%
P_{d,t}}{P_{t}}\left( 1+\tau _{d,t}\right) Y_{d,j,t}+ \\ 
+MC_{j,t}\left( \left[ \rho ^{1/\varepsilon }\left( ef_{c}Y_{c,j,t}\right)
^{\left( \varepsilon -1\right) /\varepsilon }+(1-\rho )^{1/\varepsilon
}\left( ef_{d}Y_{d,j,t}\right) ^{\left( \varepsilon -1\right) /\varepsilon }%
\right] ^{\varepsilon /(\varepsilon -1)}-Y_{t,j}\right) ,%
\end{array}%
\right\} $

for $Y_{j,t}=\left( P_{j,t}/P_{t}\right) ^{-\sigma }Y_{t}$

where $\beta ^{t}\Lambda _{0,t}=\beta ^{t}\frac{\lambda _{t}}{\lambda _{0}}$
is the real discount factor (to be clarified later...) where $MC_{j,t}$\
denotes the real marginal cost ($MC_{j,t}=MC_{j,t}^{N}/P_{t}$).

The first-order conditions are the following

FOC wrt $P_{j,t}$

$\frac{Y_{j,t}}{P_{t}}-\sigma \frac{P_{j,t}}{P_{t}^{2}}\left(
P_{j,t}/P_{t}\right) ^{-\sigma -1}Y_{t}-\gamma _{p}\left( \frac{1}{\Pi
_{t-1}^{\kappa _{p}}\Pi ^{1-\kappa _{p}}}\frac{P_{j,t}}{P_{j,t-1}}-1\right) 
\frac{Y_{t}}{P_{j,t-1}\Pi _{t-1}^{\kappa _{p}}\Pi ^{1-\kappa _{p}}}+\beta 
\mathbb{E}_{t}\Lambda _{t,t+1}\gamma _{p}\left( \frac{1}{\Pi _{t}^{\kappa
_{p}}\Pi ^{1-\kappa _{p}}}\frac{P_{j,t+1}}{P_{j,t}}-1\right) \frac{%
P_{j,t+1}Y_{t+1}}{P_{j,t}^{2}\Pi _{t}^{\kappa _{p}}\Pi ^{1-\kappa _{p}}}%
+MC_{j,t}\sigma \frac{1}{P_{t}}\left( P_{j,t}/P_{t}\right) ^{-\sigma
-1}Y_{t}=0$

Assuming symmetry and simplifying we obtain the New Keynesian Phillips curve
(NKPC):

\begin{equation*}
1-\sigma -\gamma _{p}\left( \frac{\Pi _{t}}{\Pi _{t-1}^{\kappa _{p}}\Pi
^{1-\kappa _{p}}}-1\right) \frac{\Pi _{t}}{\Pi _{t-1}^{\kappa _{p}}\Pi
^{1-\kappa _{p}}}+\beta \mathbb{E}_{t}\Lambda _{t,t+1}\gamma _{p}\left( 
\frac{\Pi _{t+1}}{\Pi _{t}^{\kappa _{p}}\Pi ^{1-\kappa _{p}}}-1\right) \frac{%
\Pi _{t+1}}{\Pi _{t}^{\kappa _{p}}\Pi ^{1-\kappa _{p}}}\frac{Y_{t+1}}{Y_{t}}%
+MC_{t}\sigma =0,
\end{equation*}

FOC wrt $Y_{c,j,t}$

$-\frac{P_{c,t}}{P_{t}}\left( 1+\tau _{c,\,t}\right) +MC_{j,t}\frac{%
\varepsilon }{\varepsilon -1}\rho ^{1/\varepsilon }\frac{\varepsilon -1}{%
\varepsilon }ef_{c}\left( ef_{c}Y_{c,j,t}\right) ^{\frac{\varepsilon -1}{%
\varepsilon }-1}\left[ \rho ^{1/\varepsilon }\left( ef_{c}Y_{c,j,t}\right)
^{\left( \varepsilon -1\right) /\varepsilon }+(1-\rho )^{1/\varepsilon
}\left( ef_{d}Y_{d,j,t}\right) ^{\left( \varepsilon -1\right) /\varepsilon }%
\right] ^{\varepsilon /(\varepsilon -1)-1}=0$

imposing symmetry:

\begin{equation*}
\frac{P_{c,t}}{P_{t}}\left( 1+\tau _{c,t}\right) =MC_{t}\rho ^{1/\varepsilon
}ef_{c}\left( ef_{c}Y_{c,t}\right) ^{\frac{\varepsilon -1}{\varepsilon }%
-1}Y_{t}^{-1}
\end{equation*}

FOC wrt $Y_{d,j,t}$

\bigskip 
\begin{equation*}
\frac{P_{d,t}}{P_{t}}\left( 1+\tau _{d,t}\right) =MC_{t}\rho ^{1/\varepsilon
}ef_{d}\left( ef_{d}Y_{d,t}\right) ^{\frac{\varepsilon -1}{\varepsilon }%
-1}Y_{t}^{-1}
\end{equation*}

Summing-up for the final good sector we keep the following equations:

\begin{itemize}
\item $Y_{t}=\left[ \rho ^{1/\varepsilon }\left( ef_{c}Y_{c,t}\right)
^{\left( \varepsilon -1\right) /\varepsilon }+(1-\rho )^{1/\varepsilon
}\left( ef_{d}Y_{d,t}\right) ^{\left( \varepsilon -1\right) /\varepsilon }%
\right] ^{\varepsilon /(\varepsilon -1)}$ \ \ (production function)

\item $1-\sigma -\gamma _{p}\left( \frac{\Pi _{t}}{\Pi _{t-1}^{\kappa
_{p}}\Pi ^{1-\kappa _{p}}}-1\right) \frac{\Pi _{t}}{\Pi _{t-1}^{\kappa
_{p}}\Pi ^{1-\kappa _{p}}}+\beta \mathbb{E}_{t}\Lambda _{t,t+1}\gamma
_{p}\left( \frac{\Pi _{t+1}}{\Pi _{t}^{\kappa _{p}}\Pi ^{1-\kappa _{p}}}%
-1\right) \frac{\Pi _{t+1}}{\Pi _{t}^{\kappa _{p}}\Pi ^{1-\kappa _{p}}}\frac{%
Y_{t+1}}{Y_{t}}+MC_{t}\sigma =0$ \ \ (NKPC)

\item $\frac{P_{c,t}}{P_{t}}\left( 1+\tau _{c,t}\right) =MC_{t}\rho
^{1/\varepsilon }ef_{c}\left( ef_{c}Y_{c,t}\right) ^{\frac{\varepsilon -1}{%
\varepsilon }-1}Y_{t}^{-1}$ \ \ \ (demand for clean input)

\item $\frac{P_{d,t}}{P_{t}}\left( 1+\tau _{d,t}\right) =MC_{t}\rho
^{1/\varepsilon }ef_{d}\left( ef_{d}Y_{d,t}\right) ^{\frac{\varepsilon -1}{%
\varepsilon }-1}Y_{t}^{-1}$ \ \ \ (demand for dirty input)

\item $PRO_{t}=Y_{t}-\frac{\gamma _{p}}{2}\left( \frac{\Pi _{t}}{\Pi
_{t-1}^{\kappa _{p}}\Pi ^{1-\kappa _{p}}}-1\right) ^{2}Y_{t}-\frac{P_{c,t}}{%
P_{t}}\left( 1+\tau _{c,t}\right) Y_{c,t}-\frac{P_{d,t}}{P_{t}}\left( 1+\tau
_{d,t}\right) Y_{d,t}$ (real profits)
\end{itemize}

\subsubsection{Intermediate sectors}

Competitive firms in the intermediate sectors transform physical capital and
labor into an intermediate good that is sold to firms in the final good
sector. There are two intermediate sectors, a clean and a dirty sector, both
producing according to Cobb-Douglas production functions:

\begin{equation*}
Y_{c,t}=A_{c,t}\left( U_{c,t}\xi _{c,t}K_{c,t}\right) ^{\alpha
_{c}}L_{c,t}^{1-\alpha _{c}},
\end{equation*}

\begin{equation*}
Y_{d,t}=A_{d,t}\left( U_{d,t}\xi _{d,t}K_{d,t}\right) ^{\alpha
_{d}}L_{d,t}^{1-\alpha _{d}},
\end{equation*}%
where $K_{x,t}$ and $L_{x,t}$ denote the amount of capital and labor used by
firms in sector $x=c,d$\ , $\alpha _{c},\alpha _{d}$ $\in (0,1)$ denote the
share of capital in the respective production function, and $U_{x,t}$ is the
utilization rate of capital.

$A_{x,t}=A_{x}e^{u_{a,x,{t}}+u_{a,t}}$ represents the sector-specific
productivity, where $A_{x}$ denotes the steady-state productivity level,
while $u_{a,x,{t}}$ and $u_{a,t}$ are two AR(1) processes capturing
sector-specific and aggregate exogenous changes in productivity.

$\xi _{x,t\text{ }}=\xi _{x\text{ }}e^{u_{\xi ,x,{t}}}$ is the quality of
capital, where: $\xi _{x\text{ }}$is the steady-state level and $u_{\xi ,x}$
is an exogenous process capturing any exogenous variation in the value of
capital able to trigger sudden variations in its market value and changes in
investment expenditure. The process $u_{\xi ,x,t}$ is such that $u_{\xi
,x,t}=\rho _{\xi ,x}u_{\xi ,x,t-1}+\varepsilon _{\xi ,x,t}$, where $0<\rho
_{\xi ,x}<1$ and $\varepsilon _{\xi ,x}\sim i.i.d.$ $N(0,\sigma _{\xi
,x}^{2}).$\ A capital quality schock is able to mimic a recession
originating from an adverse shock on the asset price.\ It directly affects
the capital in use for production and indirectly influences future
investments by changing their expected return.\ \ \ \ \ \ \ 

At the beginning of the period $t$ firms chose labor and rate of utilization
of capital to be used in production.

At the end of period $t$ intermediate goods producers purchase at price $%
Q_{x,t}$ physical capital$\ K_{x,t+1}$ to be used in production in period $%
t+1$. Firms finance their capital acquisition by obtaining funds from
financial intermediaries. Firms can obtain funds from the banks without any
financial friction by issuing claims $S_{x,t}$. The price of each claim
equal the price of a unit of capital:

\begin{equation*}
Q_{c,t}K_{c,t+1}=Q_{c,t}S_{c,t},
\end{equation*}

\begin{equation*}
Q_{d,t}K_{d,t+1}=Q_{d,t}S_{d,t}.
\end{equation*}

At the end of period $t+1$, after production, firms replaces the depreciated
capital, sell their entire capital stock and purchases capital that will be
employed in the subsequent period. The replacement price of used capital is
fixed as unity in both the sectors. The value of the stock of capital at the
end of period $t+1$ is then:

\begin{equation*}
\left( Q_{c,t+1}-\delta \left( U_{c,t+1}\right) \right) \xi
_{c,t+1}K_{c,t+1},
\end{equation*}

\begin{equation*}
\left( Q_{d,t+1}-\delta \left( U_{d,t+1}\right) \right) \xi
_{d,t+1}K_{d,t+1},
\end{equation*}

where $\delta (U_{x,t})=\delta _{0}+\frac{\delta _{1}}{1+\delta _{2}}%
U_{x,t}^{1+\delta _{2}}$.

Let $P_{x,t\text{ }}$be the price of the intermediate good. The objective
function of intermediate goods producing firm in period $t$ is equal to:

\begin{equation*}
\mathcal{L}_{0}^{clean}=\mathbb{E}_{0}\sum_{t=0}^{\infty }\beta ^{t}\Lambda
_{0,t}\left[ \frac{P_{c,t}}{P_{t}}Y_{c,t}+\left( Q_{c,t}-\delta
(U_{c,t})\right) \xi _{c,t}K_{c,t}-R_{c,t}Q_{c,t-1}K_{c,t}-W_{t}L_{c,t}%
\right]
\end{equation*}%
\begin{equation*}
\mathcal{L}_{0}^{dirty}=\mathbb{E}_{0}\sum_{t=0}^{\infty }\beta ^{t}\Lambda
_{0,t}\left[ \frac{P_{d,t}}{P_{t}}Y_{d,t}+\left( Q_{d,t}-\delta
(U_{d,t})\right) \xi _{d,t}K_{d,t}-R_{d,t}Q_{d,t-1}K_{d,t}-W_{t}L_{d,t}%
\right]
\end{equation*}

FOC wrt $L_{c,t}$

$W_{t}=\frac{P_{c,t}}{P_{t}}\left( 1-\alpha _{c}\right) \frac{Y_{c,t}}{%
L_{c,t}}$

FOC wrt $U_{c,t}$

\bigskip $\delta ^{\prime }(U_{c,t})\xi _{c,t}K_{c,t}=\frac{P_{c,t}}{P_{t}}%
\alpha _{c}\frac{Y_{c,t}}{U_{c,t}}$

The zero-profit condition implies that producers will pay the ex-post return
to capital to the intermediary. This implies that per unit of capital what
the firm pays to the intermediary (the user cost of capital) is $%
R_{k,c,t+1}Q_{t}$ and must be equal to the marginal product of capital $%
\frac{P_{c,t+1}}{P_{t+1}}\alpha _{c}\frac{Y_{c,t+1}}{K_{c,t+1}}$ plus the
value of the depreciated capital that is sold to capital producers

$R_{c,t+1}Q_{c,t}=\frac{P_{c,t+1}}{P_{t+1}}\alpha _{c}\frac{Y_{c,t+1}}{%
K_{c,t+1}}+\left( Q_{c,t+1}-\delta (U_{c,t+1})\right) \xi _{c,t+1}$

This equation can be written as

$R_{c,t+1}=\frac{\left[ \frac{P_{c,t+1}}{P_{t+1}}\alpha _{c}\frac{Y_{c,t+1}}{%
\xi _{c,t+1}K_{c,t+1}}+\left( Q_{c,t+1}-\delta (U_{c,t+1})\right) \right]
\xi _{c,t+1}}{Q_{c,t}}$

FOC wrt $L_{d,t}$

$W_{t}=\frac{P_{d,t}}{P_{t}}\left( 1-\alpha _{d}\right) \frac{Y_{d,t}}{%
L_{d,t}}$

FOC wrt $U_{d,t}$

$\delta ^{\prime }(U_{d,t})\xi _{d,t}K_{d,t}=\frac{P_{d,t}}{P_{t}}\alpha _{d}%
\frac{Y_{d,t}}{U_{d,t}}$

In the dirty sector we have

$R_{d,t+1}=\frac{\left[ \frac{P_{d,t+1}}{P_{t+1}}\alpha _{c}\frac{Y_{d,t+1}}{%
\xi _{d,t+1}K_{d,t+1}}+\left( Q_{d,t+1}-\delta (U_{d,t+1})\right) \right]
\xi _{d,t+1}}{Q_{t}}$

Equlibrium in the labor mkt

$L_{t}=L_{d,t}+L_{c,t}$

\bigskip

We keep the following equations

\begin{itemize}
\item $Y_{c,t}=A_{c,t}\left( U_{c,t}\xi _{c,t}K_{c,t}\right) ^{\alpha
_{c}}L_{c,t}^{1-\alpha _{c}}$ (production function clean sector)

\item $Y_{d,t}=A_{d,t}\left( U_{d,t}\xi _{d,t}K_{d,t}\right) ^{\alpha
_{d}}L_{d,t}^{1-\alpha _{d}}$ (production function dirty sector)

\item $W_{t}=\frac{P_{c,t}}{P_{t}}\left( 1-\alpha _{c}\right) \frac{Y_{c,t}}{%
L_{c,t}}$ (labor demand clean sector)

\item $W_{t}=\frac{P_{d,t}}{P_{t}}\left( 1-\alpha _{d}\right) \frac{Y_{d,t}}{%
L_{d,t}}$ (labor demand dirty sector)

\item $L_{t}=L_{d,t}+L_{c,t}$ (equilibrium condition in the labor market)

\item $\delta ^{\prime }(U_{c,t})\xi _{c,t}K_{c,t}=\frac{P_{c,t}}{P_{t}}%
\alpha _{c}\frac{Y_{c,t}}{U_{c,t}}$ (optimal utilization rate of capital
clean sector)

\item $\delta ^{\prime }(U_{d,t})\xi _{d,t}K_{d,t}=\frac{P_{d,t}}{P_{t}}%
\alpha _{d}\frac{Y_{d,t}}{U_{d,t}}$ (optimal utilization rate of capital
dirty sector)

\item $R_{c,t+1}=\frac{\left[ \frac{P_{c,t+1}}{P_{t+1}}\alpha _{c}\frac{%
Y_{c,t+1}}{\xi _{c,t+1}K_{c,t+1}}+\left( Q_{c,t+1}-\delta (U_{c,t+1})\right) %
\right] \xi _{c,t+1}}{Q_{c,t}}$ (asset price equation clean sector)

\item $R_{d,t+1}=\frac{\left[ \frac{P_{d,t+1}}{P_{t+1}}\alpha _{d}\frac{%
Y_{d,t+1}}{\xi _{d,t+1}K_{d,t+1}}+\left( Q_{d,t+1}-\delta (U_{d,t+1})\right) %
\right] \xi _{d,t+1}}{Q_{d,t}}$ (asset price equation dirty sector)

\item $Q_{c,t}K_{c,t+1}=Q_{c,t}S_{c,t}$ (arbitrage condition clean sector)

\item $Q_{d,t}K_{d,t+1}=Q_{d,t}S_{d,t}$ (arbitrage condition dirty sector)
\end{itemize}

Plus the processes for deltas, $\xi $ and $A$.

\subsubsection{Capital producers}

At the end of the period, after the production of intermediate goods is
finalized, competitive capital producers in sector $x=c,d$\ \ buy the
depreciated capital stock from intermediate good producers, rebuild
depreciated capital\ and build new capital. They then sell both the new and
refurbished capital to intermediate good producers at price $Q_{x,t}$.

Capital evolves according to:

\begin{equation*}
K_{c,t+1}=\xi _{c,t}K_{c,t}+I_{net,c,t}
\end{equation*}

\begin{equation*}
K_{d,t+1}=\xi _{d,t}K_{d,t}+I_{net,d,t}
\end{equation*}

\bigskip

where $I_{net,x,t}$ are net investment (i.e new capital) defined as the
difference between the gross capital created $I_{x,t}$ and the amount of
capital refurbished $\delta (U_{x,t})\xi _{x,t}K_{x,t}$:

\begin{equation*}
I_{net,c,t}=I_{c,t}-\delta (U_{c,t})\xi _{c,t}K_{c,t}
\end{equation*}

\begin{equation*}
I_{net,d,t}=I_{d,t}-\delta (U_{d,t})\xi _{d,t}K_{d,t}
\end{equation*}

\bigskip

The cost capital producers incur to replace capital is unity, but there are
adjustment cost associated to the new capital production.\textbf{\ }The
discounted profits function for capital producers is the following:

\begin{equation*}
\max \mathbb{E}_{t}\dsum\limits_{\tau =t}^{\infty }\beta ^{\tau -t}\Lambda
_{t,t+\tau }\left\{ (Q_{c,\tau }-1)I_{net,c,\tau }-f\left( \frac{%
I_{net,c,\tau }+I_{c}}{I_{net,c,\tau -1}+I_{c}}\right) \left( I_{net,c,\tau
}+I_{c}\right) \right\} ,
\end{equation*}

\begin{equation*}
\max \mathbb{E}_{t}\dsum\limits_{\tau =t}^{\infty }\beta ^{\tau -t}\Lambda
_{t,t+\tau }\left\{ (Q_{d,\tau k}-1)I_{net,d,\tau }-f\left( \frac{%
I_{net,d,\tau }+I_{d}}{I_{net,d,\tau -1}+I_{d}}\right) \left( I_{net,d,\tau
}+I_{d}\right) \right\} ,
\end{equation*}

with $f=\frac{\gamma _{i,c}}{2}\left( \frac{I_{net,c,t}+I_{c}}{%
I_{net,c,t-1}+I_{c}}-1\right) ^{2}$and $I_{c}$,$I_{d}$ are the steady state
investments.

FOC wrt $I_{net,c,\tau }$

$Q_{c,t}=1+\frac{\gamma _{i,c}}{2}\left( \frac{I_{net,c,t}+I_{c}}{%
I_{net,c,t-1}+I_{c}}-1\right) ^{2}+\gamma _{i,c}\left( \frac{%
I_{net,c,t}+I_{c}}{I_{net,c,t-1}+I_{c}}-1\right) \frac{I_{net,c,t}+I_{c}}{%
I_{net,c,t-1}+I_{c}}-\mathbb{E}_{t}\beta \Lambda _{t,t+1}\gamma _{i,c}\left( 
\frac{I_{net,c,t+1}+I_{c}}{I_{net,c,t}+I_{c}}-1\right) \left( \frac{%
I_{net,c,t+1}+I_{c}}{I_{net,c,t}+I_{c}}\right) ^{2}$

FOC wrt $I_{net,d,\tau }$

$Q_{d,t}=1+\frac{\gamma _{i,d}}{2}\left( \frac{I_{net,d,t}+I_{d}}{%
I_{net,d,t-1}+I_{d}}-1\right) ^{2}+\gamma _{i,d}\left( \frac{%
I_{net,d,t}+I_{d}}{I_{net,d,t-1}+I_{d}}-1\right) \frac{I_{net,d,t}+I_{d}}{%
I_{net,d,t-1}+I_{d}}-\mathbb{E}_{t}\beta \Lambda _{t,t+1}\gamma _{i,d}\left( 
\frac{I_{net,d,t+1}+I_{d}}{I_{net,d,t}+I_{d}}-1\right) \left( \frac{%
I_{net,d,t+1}+I_{d}}{I_{net,d,t}+I_{d}}\right) ^{2}$

From the capital producer sector we keep the following equations

\begin{itemize}
\item $K_{c,t+1}=\xi _{c,t}K_{c,t}+I_{net,c,t}$ (evolution of capital in the
clean sector)

\item $K_{d,t+1}=\xi _{d,t}K_{d,t}+I_{net,d,t}$ (evolution of capital in the
dirty sector)

\item $I_{net,c,t}=I_{c,t}-\delta (U_{c,t})\xi _{c,t}K_{c,t}$ (evolution of
net investment in the clean sector)

\item $I_{net,d,t}=I_{d,t}-\delta (U_{d,t})\xi _{d,t}K_{d,t}$ (evolution of
net investment in the dirty sector)

\item $Q_{c,t}=1+\frac{\gamma _{i,c}}{2}\left( \frac{I_{net,c,t}+I_{c}}{%
I_{net,c,t-1}+I_{c}}-1\right) ^{2}+\gamma _{i,c}\left( \frac{%
I_{net,c,t}+I_{c}}{I_{net,c,t-1}+I_{c}}-1\right) \frac{I_{net,c,t}+I_{c}}{%
I_{net,c,t-1}+I_{c}}-\mathbb{E}_{t}\beta \Lambda _{t,t+1}\gamma _{i,c}\left( 
\frac{I_{net,c,t+1}+I_{c}}{I_{net,c,t1}+I_{c}}-1\right) \left( \frac{%
I_{net,c,+1}+I_{c}}{I_{net,c,t}+I_{c}}\right) ^{2}$ (price of unit of
capital in the clean sector)

\item $Q_{d,t}=1+\frac{\gamma _{i,d}}{2}\left( \frac{I_{net,d,t}+I_{d}}{%
I_{net,d,t-1}+I_{d}}-1\right) ^{2}+\gamma _{i,d}\left( \frac{%
I_{net,d,t}+I_{d}}{I_{net,d,t-1}+I_{d}}-1\right) \frac{I_{net,d,t}+I_{d}}{%
I_{net,d,t-1}+I_{d}}-\mathbb{E}_{t}\beta \Lambda _{t,t+1}\gamma _{i,d}\left( 
\frac{I_{net,d,t+1}+I_{d}}{I_{net,d,t1}+I_{d}}-1\right) \left( \frac{%
I_{net,d,+1}+I_{d}}{I_{net,d,t}+I_{d}}\right) ^{2}$ (price of unit of
capital in the dirty sector)
\end{itemize}

\bigskip

\subsection{Financial intermediaries}

Financial intermerdiaries lend funds obtained from households to
intermediate-good producers. At the end of period $t$ the typical financial
intermediary's balance sheet is equal to:%
\begin{equation}
Q_{c,t}S_{c,j,t}+Q_{d,t}S_{d,j,t}=N_{j,t}+B_{j,t}  \label{balance sheet}
\end{equation}%
where $N_{t}$ is the net worth (bank equity capital). In $t+1$ the
representative financial intermediary earns the stochastic returns $%
R_{c,k,t+1}$ and $R_{d,k,t+1}$, and pays the non-contingent nominal gross
return $R_{t}$ on deposits. Returns on loans are sector specific: they
depend on the price of capital, on the payoffs, and on the capital quality
shocks.

The evolution of net worth over time is thus given by the difference between
earnings on assets and interest payments on liabilities:%
\begin{equation}
N_{j,t+1}=R_{c,t+1}Q_{c,t}S_{c,j,t}+R_{d,t+1}Q_{d,t}S_{d,j,t}-R_{t}B_{j,t}
\label{net_worth_over_time}
\end{equation}

Using the definition of $B_{j,t+1}$ that we have in eq. \ref{balance sheet}
we obtain:%
\begin{equation*}
N_{j,t+1}=(R_{c,t+1}-R_{t})Q_{c,t}S_{c,j,t}+(R_{d,t+1}-R_{t})Q_{d,t}S_{d,j,t}+R_{t}N_{j,t}
\end{equation*}%
where $R_{c,t+1}-R_{t}$ and $R_{d,t+1}-R_{t}$ are the risk premia on clean
and dirty assets that determine the growth in bank's wealth above the
riskless return. Note that with perfect capital markets the risk premium
would be zero ($R_{c,t+1}=R_{d,t+1}=R_{t})$. In this economy, risk premia
are positive, due to the limited intermediary's ability to obtain funds.

The banker will fund assets as long as%
\begin{equation*}
E_{t}\beta ^{i}\Lambda _{t,t+1+i}(R_{c,t+1+i}-R_{t+i})\geq 0,\text{ }i\geq 0
\end{equation*}%
\begin{equation*}
E_{t}\beta ^{i}\Lambda _{t,t+1+i}(R_{d,t+1+i}-R_{t+i})\geq 0,\text{ }i\geq 0
\end{equation*}%
where $\beta ^{i}\Lambda _{t,t+1+i}$ is the stochastic discount factor.

Let $\Xi _{j,t}\equiv Q_{c,t}S_{c,j,t}+Q_{d,t}S_{d,j,t}$. Rewriting the
equation for the evolution of the net worth we obtain: (\textbf{FRA: ci
serve poi per esprimere }$N_{j,t+1}$\textbf{\ in termini di leverage})

$%
N_{j,t+1}=(R_{c,t+1}-R_{t})Q_{c,t}S_{c,j,t}+(R_{d,t+1}-R_{t})Q_{d,t}S_{d,j,t}+R_{t}N_{j,t}= 
$

$\ \ \ \ \ \ \ \
=(R_{c,t+1}-R_{t})Q_{c,t}S_{c,j,t}+R_{d,t+1}Q_{d,t}S_{d,j,t}-R_{t}Q_{d,t}S_{d,j,t}+R_{c,t+1}Q_{d,t}S_{d,j,t}-R_{c,t+1}Q_{d,t}S_{d,j,t}+R_{t}N_{j,t}= 
$

\ \ \ \ \ \ \ $%
=(R_{c,t+1}-R_{t})Q_{c,t}S_{c,j,t}-Q_{d,t}S_{d,j,t}(R_{c,t+1}-R_{d,t+1})+(R_{c,t+1}-R_{t})Q_{d,t}S_{d,j,t}+R_{t}N_{j,t}= 
$

\ \ \ \ \ \ $\
=(R_{c,t+1}-R_{t})(Q_{c,t}S_{c,j,t}+Q_{d,t}S_{d,j,t})-Q_{d,t}S_{d,j,t}(R_{c,t+1}-R_{d,t+1})+R_{t}N_{j,t}= 
$

\ \ \ \ \ \ \ $=\left[ (R_{c,t+1}-R_{t})-\frac{Q_{d,t}S_{d,j,t}}{\Xi _{t}}%
(R_{c,t+1}-R_{d,t+1})\right] \Xi _{t}+R_{t}N_{j,t}$

where $(R_{c,t+1}-R_{d,t+1})$ is the spread between clean and dirty assets.

\begin{equation*}
N_{j,t+1}=\left[ (R_{c,t+1}-R_{t})-\frac{Q_{d,t}S_{d,j,t}}{\Xi _{j,t}}%
(R_{c,t+1}-R_{d,t+1})\right] \Xi _{j,t}+R_{t}N_{j,t}
\end{equation*}

The banker's objective is to maximize expected terminal wealth (remind:
bankers have a finite horizon!), given by:

\begin{eqnarray}
V_{j,t} &=&\max \mathbb{E}_{t}\sum_{i=0}^{\infty }(1-\theta )\theta
^{i}\beta ^{i+1}\Lambda _{t,t+1+i}\left( N_{j,t+1+i}\right)
\label{terminal wealth} \\
&=&\max \mathbb{E}_{t}\sum_{i=0}^{\infty }(1-\theta )\theta ^{i}\beta
^{i+1}\Lambda _{t,t+1+i}\left\{ \left[ (R_{c,t+1+i}-R_{t+i})\Xi
_{j,t+i}-Q_{d,t+i}S_{d,j,t+i}(R_{c,t+1+i}-R_{d,t+1+i})\right]
+R_{t+i}N_{j,t+i}\right\}  \notag
\end{eqnarray}

In order to limit the ability of the intermediary to expand its assets
indefinitely, a moral hazard enforcement problem (derived from agency cost)
is assumed: at the beginning of each period the banker can choose to divert
a fraction $0<\rho <1$ of available funds from the banking activity. If the
intermediary does not honor its debt, households (depositors) can liquidate
the intermediate, by forcing it into bankruptcy and obtain the fraction $1-$ 
$\rho $ of initial assets, or can limit the funds they lend to banks. The
incentive constraint faced by households to be willing to supply funds to
the banker is:%
\begin{equation*}
V_{j,t}\geq \rho (Q_{c,t}S_{c,j,t}+Q_{d,t}S_{d,j,t})
\end{equation*}%
where $V_{j,t}$ represents the loss supported by the banker when he diverts
a fraction of assets (i.e. the expected discounted value of its terminal
wealth) and $\rho (Q_{c,t}S_{c,j,t}+Q_{d,t}S_{d,j,t})$ is the gain from
diverting funds. We assume that the fraction $\rho $\ which intermediaries
can divert is the same across sectors. (\textbf{FRA: va argomentato. Se
fosse diverso forse "gli households presterebbero al settore" con il }$\rho $
\textbf{piu' basso? BB: La grandezza di rho si basa su quanto possano
"rubare" i banchieri, per questo mi sembra difficile giustificarne uno
diverso per settore.})

We rewrite recursively the objective of the bank as follows:%
\begin{equation}
V_{j,t}=\max \beta E_{t}\left\{ \Lambda _{t,t+1}\left[ \left( 1-\theta
\right) N_{j,t+1}+\theta V_{j,t+1}\right] \right\}  \label{bank_objective}
\end{equation}

and we assume the following linear solution:%
\begin{eqnarray*}
V_{j,t} &=&\nu _{t}\Xi _{j,t}+\eta _{t}N_{j,t} \\
&&\nu _{t}\left( Q_{c,t}S_{c,j,t}+Q_{d,t}S_{d,j,t}\right) +\eta _{t}N_{j,t}
\end{eqnarray*}

\textbf{\bigskip }

where $\nu _{t}$\ is the expected discounted marginal gain of an extra unit
of assets $\Xi _{j,t}=Q_{c,t}S_{c,j,t}+Q_{d,t}S_{d,j,t}$, and $\eta _{t}$\
is the expected discounted marginal gain of an extra unit of net worth $%
N_{j,t}$. The incentive constraint becomes:

\begin{eqnarray*}
\nu _{t}\Xi _{j,t}+\eta _{t}N_{j,t} &\geq &\rho \Xi _{j,t} \\
\nu _{t}(Q_{c,t}S_{c,j,t}+Q_{d,t}S_{d,j,t})+\eta _{t}N_{j,t} &\geq &\rho
(Q_{c,t}S_{c,j,t}+Q_{d,t}S_{d,j,t})
\end{eqnarray*}

if binding:

\begin{eqnarray*}
\Xi _{j,t} &=&\frac{\eta _{t}}{\rho -\nu _{t}}N_{j,t} \\
Q_{c,t}S_{c,j,t}+Q_{d,t}S_{d,j,t} &=&\frac{\eta _{t}}{\rho -\nu _{t}}N_{j,t}
\end{eqnarray*}

where\ $\frac{\eta _{t}}{\rho -\nu _{t}}$\ is the private leverage ratio.
Denoting $\phi _{t}=\frac{\eta _{t}}{\rho -\nu _{t}}$ we obtain:

\begin{equation*}
Q_{c,t}S_{c,j,t}+Q_{d,t}S_{d,j,t}=\phi _{t}N_{j,t}
\end{equation*}%
\begin{equation*}
\Xi _{j,t}=\phi _{t}N_{j,t}
\end{equation*}

Substituting into (\ref{bank_objective})

$V_{j,t}=\max \beta \mathbb{E}_{t}\left\{ \Lambda _{t,t+1}\left[ \left(
1-\theta \right) N_{j,t+1}+\theta (\nu _{t+1}\Xi _{j,t+1}+\eta
_{t+1}N_{j,t+1})\right] \right\} $

recalling $\Xi _{j,t}=\phi _{t}N_{j,t}$

$V_{j,t}=\max \beta \mathbb{E}_{t}\left\{ \Lambda _{t,t+1}\left[ \left(
1-\theta \right) N_{j,t+1}+\theta (\nu _{t+1}\phi _{t+1}N_{j,t+1}+\eta
_{t+1}N_{j,t+1})\right] \right\} $

$V_{j,t}=\max \beta \mathbb{E}_{t}\left\{ \Lambda _{t,t+1}\left[ 1+\theta
(\nu _{t+1}\phi _{t+1}+\eta _{t+1}-1)\right] N_{j,t+1}\right\} $

optimizing respect to $N_{j,t+1}$ we obtain the banker effective discount
factor:

$\Omega _{j,t+1}=\beta \Lambda _{t,t+1}\left[ 1+\theta (\nu _{t+1}\phi
_{t+1}+\eta _{t+1}-1)\right] $

we can drop the j-index

$\Omega _{t+1}=\beta \Lambda _{t,t+1}\left[ 1+\theta (\nu _{t+1}\phi
_{t+1}+\eta _{t+1}-1)\right] $

Recalling $V_{j,t}=\max \mathbb{E}_{t}\sum_{i=0}^{\infty }(1-\theta )\theta
^{i}\beta ^{i+1}\Lambda _{t,t+1+i}\left\{ \left[ (R_{c,t+1+i}-R_{t+i})\Xi
_{j,t+i}-Q_{d,t+i}S_{d,j,t+i}(R_{c,t+1+i}-R_{d,t+1+i})\right]
+R_{t+i}N_{j,t+i}\right\} $

$\mathbb{E}_{t}\left[ \Omega _{t+1}(R_{c,t+1}-R_{t})\right] =\nu _{t}$ \ $\
\ \ \ \ \ \ \ \ \ \ \ \ \ \nu _{t}>0$

$\mathbb{E}_{t}\left( \Omega _{t+1}R_{t}\right) =\eta _{t}$ \ \ \ \ \ \ \ \
\ \ \ \ \ \ \ \ \ \ \ \ \ \ \ \ \ \ \ \ $\eta _{t}>1$

the FOC wrt $s_{d,t}=Q_{d,t}S_{d,j,t}/\Xi _{j,t}$

$\mathbb{E}_{t}\left[ \Omega _{t+1}(R_{c,t+1}-R_{d,t+1})\right] =0$

we can choose $s_{d,t}$ so that $\mathbb{E}_{t}\left[ \Omega
_{t+1}(R_{c,t+1}-R_{d,t+1})\right] =0$ holds true up to a second-order
approximation.

Note that in an economy without financial frictions we would have $\eta
_{t}=1$ and $\nu _{t}=0$ (i.e. the discount factor of the bank would then be 
$\Omega _{j,t+1}=\beta \Lambda _{t,t+1}(1-\theta )$)

Rewriting the equation for the evolution of the net worth we obtain,
dropping the j-index:

\begin{equation*}
N_{t+1}=\left\{ \left[ (R_{c,t+1}-R_{t})-\frac{Q_{d,t}S_{d,t}}{\Xi _{t}}%
(R_{c,t+1}-R_{d,t+1})\right] \phi _{t}+R_{t}\right\} N_{t}
\end{equation*}

All the component of $\phi _{t}$ do not depend on firm-specific factors. We
have:%
\begin{equation*}
Q_{c,t}S_{c,t}+Q_{d,t}S_{d,t}=\phi _{t}N_{t}
\end{equation*}

describing the total intermediary demand for assets and the aggregate
quantity of intermediary equity capital.

Total net worth in the banking sector equal the sum of the net worth of
existing bankers $N_{e,t}$\ and that of new bankers $N_{n,t}$:

\begin{equation*}
N_{t}=N_{e,t}+N_{n,t}
\end{equation*}

Each period household transfers a start up fund to the new bankers. This
fund is equal to a fraction $\omega /(1-\theta )$ of the value of assets of
intermediaries who exit $(1-\theta )(Q_{c,t}S_{c,t-1}+Q_{d,t}S_{d,t-1})$:

\begin{equation*}
N_{n,t}=\omega (Q_{c,t}S_{c,t-1}+Q_{d,t}S_{d,t-1})
\end{equation*}

\bigskip The net worth for the existing bankers is:

\begin{equation*}
N_{e,t}=\theta \left\{ \left[ (R_{c,t}-R_{t-1})-\frac{Q_{d,t-1}S_{d,j,t-1}}{%
\Xi _{t-1}}(R_{c,t}-R_{d,t})\right] \phi _{t-1}+R_{t-1}\right\} N_{t-1}
\end{equation*}

Equations of the banking sector to be used to solve the model

\begin{itemize}
\item $\Xi _{t}=N_{t}+B_{t}$ (balance sheet)

\item $N_{t}=N_{e,t}+N_{n,t}$ (total net worth)

\item $N_{n,t}=\omega (Q_{c,t}S_{c,t-1}+Q_{d,t}S_{d,t-1})$ (net worth of new
bankers)

\item $N_{e,t}=\theta \left\{ \left[ (R_{c,t}-R_{t-1})-\frac{%
Q_{d,t-1}S_{d,j,t-1}}{\Xi _{t-1}}(R_{c,t}-R_{d,t})\right] \phi
_{t-1}+R_{t-1}\right\} N_{t-1}$ (net worth of existing bankers)

\item $\Xi _{t}=Q_{c,t}S_{c,t}+Q_{d,t}S_{d,t}$ (total value of loans)

\item $Q_{c,t}S_{c,t}+Q_{d,t}S_{d,t}=\phi _{t}N_{t}$ (incentive constraint)

\item $\phi _{t}=\frac{\eta _{t}}{\rho -\nu _{t}}$ (private leverage ratio)

\item $\Omega _{t+1}=\beta \Lambda _{t,t+1}\left[ 1+\theta (\nu _{t+1}\phi
_{t+1}+\eta _{t+1}-1)\right] $ (effective discount factor of bankers)

\item $\mathbb{E}_{t}\left[ \Omega _{t+1}(R_{c,t+1}-R_{t})\right] =\nu _{t}$

\item $\mathbb{E}_{t}\left( \Omega _{t+1}R_{t}\right) =\eta _{t}$

\item $\mathbb{E}_{t}\left[ \Omega _{t+1}(R_{c,t+1}-R_{d,t+1})\right]
=0\rightarrow s_{d,t}=Q_{d,t}S_{d,j,t}/\Xi _{j,t}$ \ \ \ (we should be able
to define $s_{d,t}$ so that the $\mathbb{E}_{t}\left[ \Omega
_{t+1}(R_{c,t+1}-R_{d,t+1})\right] =0$ up to a second-order approximation...
in that case it's constant, while the rest of the model is solved linearly)

\item $V_{t}=\nu _{t}\Xi _{t}+\eta _{t}N_{t}$ (bank objective function)
\end{itemize}

\bigskip

\subsection{Fiscal and monetary authorities}

Government expenditures consist of public consumption $G_{t}$, subsidies for
the adoption of clean inputs, interest on bonds issued in the previous
period (\textbf{ricordarsi di aggiungere monitoring costs and credit policy}%
). Expenditures are fully financed by lump-sum taxes, taxes on labor income,
taxes on the use of dirty inputs, and government bonds. \textbf{FRA: ho
assunto balanced budget}

The flow budget constraint of the government is then:

\begin{equation*}
G_{t}=T_{t}+\tau _{W,t}W_{t}L_{t}+\tau _{d,t}\frac{P_{d,t}}{P_{t}}%
Y_{d,t}-\tau _{c,t}\frac{P_{c,t}}{P_{t}}Y_{c,t}.
\end{equation*}

Monetary policy is characterize by a Taylor-type interest rate rule
specified as follows:

\begin{equation*}
1+i_{t}=(1+i_{t-1})^{\kappa _{i}}\left[ (1+i)\Pi _{t}^{\kappa _{\pi
}}+\left( \frac{Y_{t}}{Y}\right) ^{\kappa _{y}}\right] ^{1-\kappa _{i}},
\end{equation*}

where $i_{t}$ denotes the nominal interest rate, $i$ the steady state value
of the nominal interest rate, $\kappa _{i}\in (0,1)$ is the smoothing
parameter, and $\kappa _{\pi }$ and $\kappa _{y}$\ capture the
responsiveness of nominal interest rate to movements in inflation and output
respectively. $Y$ denotes the steady sate level of output (\textbf{in GK the
flexible price equilibrium} \textbf{level}).

The nominal interest rate affects the real economy via the Fisher equation:

\begin{equation*}
1+i_{t}=R_{t+1}\mathbb{E}_{t}(\Pi _{t+1})
\end{equation*}

\subsection{Aggregate resource constraint}

Final output is: (\textbf{Aggiungere: expenditures on government
intermediation})

\begin{equation*}
Y_{t}=C_{t}++G_{t}+I_{c,t}+I_{d,t}+\frac{\gamma _{i,c}}{2}\left( \frac{%
I_{net,c,t}+I_{c}}{I_{net,c,t-1}+I_{c}}-1\right) ^{2}\left( I_{net,c,\tau
}+I_{c}\right) +\frac{\gamma _{i,d}}{2}\left( \frac{I_{net,d,t}+I_{d}}{%
I_{net,d,t-1}+I_{d}}-1\right) ^{2}\left( I_{net,d,\tau }+I_{d}\right) +\frac{%
\gamma _{p}}{2}\left( \frac{\Pi _{t}}{\Pi _{t-1}^{\kappa _{p}}\Pi ^{1-\kappa
_{p}}}-1\right) ^{2}Y_{t}.
\end{equation*}

\bigskip

\subsection{Credit policy (to be added)}

\bigskip

\begin{center}
**********************************************************
\end{center}

Let $\frac{P_{c,t}}{P_{t}}=p_{c,t}$ and $\frac{P_{d,t}}{P_{t}}=p_{d,t}$.

We solved the model for a decentralized economy.

\textbf{Number of equations: 43}

\begin{enumerate}
\item $\frac{1}{C_{t}-h_{c}\overset{\_}{C}_{t-1}}=\lambda _{t}$ (marginal
utility of consumption)

\item $\chi _{l}L_{t}^{\varphi _{l}}=\lambda _{t}W_{t}(1-\tau _{W,t})$
(labor supply)

\item $\beta \mathbb{E}_{t}\Lambda _{t,t+1}R_{t}=1$ \ (Euler's equation)

\item $\Lambda _{t,t+1}=\frac{\lambda _{t+1}}{\lambda _{t}}$ \ \ \ \
(discount factor)

\item $U_{t}=\ln \left( C_{t+i}-h_{c}\overset{\_}{C}_{t+i-1}\right) -\chi
_{l}\frac{L_{t+i}^{1+\varphi _{l}}}{1+\varphi _{l}}+\mathbb{E}_{t}\beta
U_{t+1}$ \ \ (welfare measure)

\item $Y_{t}=\left[ \rho ^{1/\varepsilon }\left( ef_{c}Y_{c,t}\right)
^{\left( \varepsilon -1\right) /\varepsilon }+(1-\rho )^{1/\varepsilon
}\left( ef_{d}Y_{d,t}\right) ^{\left( \varepsilon -1\right) /\varepsilon }%
\right] ^{\varepsilon /(\varepsilon -1)}$ \ \ (production function)

\item $1-\sigma -\gamma _{p}\left( \frac{\Pi _{t}}{\Pi _{t-1}^{\kappa
_{p}}\Pi ^{1-\kappa _{p}}}-1\right) \frac{\Pi _{t}}{\Pi _{t-1}^{\kappa
_{p}}\Pi ^{1-\kappa _{p}}}+\beta \mathbb{E}_{t}\Lambda _{t,t+1}\gamma
_{p}\left( \frac{\Pi _{t+1}}{\Pi _{t}^{\kappa _{p}}\Pi ^{1-\kappa _{p}}}%
-1\right) \frac{\Pi _{t+1}}{\Pi _{t}^{\kappa _{p}}\Pi ^{1-\kappa _{p}}}\frac{%
Y_{t+1}}{Y_{t}}+MC_{t}\sigma =0$ \ \ (NKPC)

\item $p_{c,t}\left( 1+\tau _{c,t}\right) =MC_{t}\rho ^{1/\varepsilon
}ef_{c}\left( ef_{c}Y_{c,t}\right) ^{\frac{\varepsilon -1}{\varepsilon }%
-1}Y_{t}^{-1}$ \ \ \ (demand for clean input)

\item $p_{d,t}\left( 1+\tau _{d,t}\right) =MC_{t}\rho ^{1/\varepsilon
}ef_{d}\left( ef_{d}Y_{d,t}\right) ^{\frac{\varepsilon -1}{\varepsilon }%
-1}Y_{t}^{-1}$ \ \ \ (demand for dirty input)

\item $Y_{c,t}=A_{c,t}\left( U_{c,t}\xi _{c,t}K_{c,t}\right) ^{\alpha
_{c}}L_{c,t}^{1-\alpha _{c}}$ (production function clean sector)

\item $Y_{d,t}=A_{d,t}\left( U_{d,t}\xi _{d,t}K_{d,t}\right) ^{\alpha
_{d}}L_{d,t}^{1-\alpha _{d}}$ (production function dirty sector)

\item $W_{t}=p_{c,t}\left( 1-\alpha _{c}\right) \frac{Y_{c,t}}{L_{c,t}}$
(labor demand clean sector)

\item $W_{t}=p_{d,t}\left( 1-\alpha _{d}\right) \frac{Y_{d,t}}{L_{d,t}}$
(labor demand dirty sector)

\item $L_{t}=L_{d,t}+L_{c,t}$ (equilibrium condition in the labor market)

\item $\delta ^{\prime }(U_{c,t})\xi _{c,t}K_{c,t}=p_{c,t}\alpha _{c}\frac{%
Y_{c,t}}{U_{c,t}}$ (optimal utilization rate of capital clean sector) $\ \ \
\ \ \ \ \ $Remember to substitute: $\delta (U_{c,t})=\delta _{0}+\frac{%
\delta _{1}}{1+\delta _{2}}U_{c,t}^{1+\delta _{2}}$ differenti $\delta $ per
i due settori?

\item $\delta ^{\prime }(U_{d,t})\xi _{d,t}K_{d,t}=p_{d,t}\alpha _{d}\frac{%
Y_{d,t}}{U_{d,t}}$ (optimal utilization rate of capital dirty sector) \ \ \
\ \ \ \ \ Remember to substitute: $\delta (U_{d,t})=\delta _{0}+\frac{\delta
_{1}}{1+\delta _{2}}U_{d,t}^{1+\delta _{2}}$

\item $R_{c,t+1}=\frac{\left[ p_{c,t+1}\alpha _{c}\frac{Y_{c,t+1}}{\xi
_{c,t+1}K_{c,t+1}}+\left( Q_{c,t+1}-\delta (U_{c,t+1})\right) \right] \xi
_{c,t+1}}{Q_{c,t}}$ (asset price equation clean sector)

\item $R_{d,t+1}=\frac{\left[ p_{d,t+1}\alpha _{d}\frac{Y_{d,t+1}}{\xi
_{d,t+1}K_{d,t+1}}+\left( Q_{d,t+1}-\delta (U_{d,t+1})\right) \right] \xi
_{d,t+1}}{Q_{d,t}}$ (asset price equation dirty sector)

\item $Q_{c,t}K_{c,t+1}=Q_{c,t}S_{c,t}$ (arbitrage condition clean sector)

\item $Q_{d,t}K_{d,t+1}=Q_{d,t}S_{d,t}$ (arbitrage condition dirty sector)

\item $K_{c,t+1}=\xi _{c,t}K_{c,t}+I_{net,c,t}$ (evolution of capital in the
clean sector)

\item $K_{d,t+1}=\xi _{d,t}K_{d,t}+I_{net,d,t}$ (evolution of capital in the
dirty sector)

\item $I_{net,c,t}=I_{c,t}-\delta (U_{c,t})\xi _{c,t}K_{c,t}$ (evolution of
net investment in the clean sector) $\ \ \ \ \ \ $Remember to substitute: $%
\delta (U_{c,t})=\delta _{0}+\frac{\delta _{1}}{1+\delta _{2}}%
U_{c,t}^{1+\delta _{2}}$

\item $I_{net,d,t}=I_{d,t}-\delta (U_{d,t})\xi _{d,t}K_{d,t}$ (evolution of
net investment in the dirty sector) \ \ \ \ \ \ Remember to substitute: $%
\delta (U_{d,t})=\delta _{0}+\frac{\delta _{1}}{1+\delta _{2}}%
U_{d,t}^{1+\delta _{2}}$

\item $Q_{c,t}=1+\frac{\gamma _{i,c}}{2}\left( \frac{I_{net,c,t}+I_{c}}{%
I_{net,c,t-1}+I_{c}}-1\right) ^{2}+\gamma _{i,c}\left( \frac{%
I_{net,c,t}+I_{c}}{I_{net,c,t-1}+I_{c}}-1\right) \frac{I_{net,c,t}+I_{c}}{%
I_{net,c,t-1}+I_{c}}-\mathbb{E}_{t}\beta \Lambda _{t,t+1}\gamma _{i,c}\left( 
\frac{I_{net,c,t+1}+I_{c}}{I_{net,c,t}+I_{c}}-1\right) \left( \frac{%
I_{net,c,t+1}+I_{c}}{I_{net,c,t}+I_{c}}\right) ^{2}$ (price of unit of
capital in the clean sector)

\item $Q_{d,t}=1+\frac{\gamma _{i,d}}{2}\left( \frac{I_{net,d,t}+I_{d}}{%
I_{net,d,t-1}+I_{d}}-1\right) ^{2}+\gamma _{i,d}\left( \frac{%
I_{net,d,t}+I_{d}}{I_{net,d,t-1}+I_{d}}-1\right) \frac{I_{net,d,t}+I_{d}}{%
I_{net,d,t-1}+I_{d}}-\mathbb{E}_{t}\beta \Lambda _{t,t+1}\gamma _{i,d}\left( 
\frac{I_{net,d,t+1}+I_{d}}{I_{net,d,t1}+I_{d}}-1\right) \left( \frac{%
I_{net,d,t+1}+I_{d}}{I_{net,d,t}+I_{d}}\right) ^{2}$ (price of unit of
capital in the dirty sector)

\item $\Xi _{t}=N_{t}+B_{t}$ (balance sheet)

\item $N_{t}=N_{e,t}+N_{n,t}$ (total net worth)

\item $N_{n,t}=\omega (Q_{c,t}S_{c,t-1}+Q_{d,t}S_{d,t-1})$ (net worth of new
bankers)

\item $N_{e,t}=\theta \left\{ \left[ (R_{c,t}-R_{t-1})-\frac{%
Q_{d,t-1}S_{d,j,t-1}}{\Xi _{t-1}}(R_{c,t}-R_{d,t})\right] \phi
_{t-1}+R_{t-1}\right\} N_{t-1}$ (net worth of existing bankers)

\item $\Xi _{t}=Q_{c,t}S_{c,t}+Q_{d,t}S_{d,t}$ (total value of loans)

\item $Q_{c,t}S_{c,t}+Q_{d,t}S_{d,t}=\phi _{t}N_{t}$ (incentive constraint)

\item $\phi _{t}=\frac{\eta _{t}}{\rho -\nu _{t}}$ (private leverage ratio)

\item $\Omega _{t+1}=\beta \Lambda _{t,t+1}\left[ 1+\theta (\nu _{t+1}\phi
_{t+1}+\eta _{t+1}-1)\right] $ (effective discount factor of bankers)

\item $\mathbb{E}_{t}\left[ \Omega _{t+1}(R_{c,t+1}-R_{t})\right] =\nu _{t}$

\item $\mathbb{E}_{t}\left( \Omega _{t+1}R_{t}\right) =\eta _{t}$

\item $\mathbb{E}_{t}\left[ \Omega _{t+1}(R_{c,t+1}-R_{d,t+1})\right]
=0\rightarrow s_{d,t}=Q_{d,t}S_{d,j,t}/\Xi _{j,t}$ \ \ \ (we should be able
to define $s_{d,t}$ so that the $\mathbb{E}_{t}\left[ \Omega
_{t+1}(R_{c,t+1}-R_{d,t+1})\right] =0$ up to a second-order approximation...
in that case it's constant, while the rest of the model is solved linearly)

\item $V_{t}=\nu _{t}\Xi _{t}+\eta _{t}N_{t}$ (bank objective function)

\item $G_{t}=T_{t}+\tau _{W,t}W_{t}L_{t}+\tau _{d,t}p_{d,t}Y_{d,t}-\tau
_{c,t}p_{c,t}Y_{c,t}$ (government budget constraint)

\item $1+i_{t}=(1+i_{t-1})^{\kappa _{i}}\left[ (1+i)\Pi _{t}^{\kappa _{\pi
}}+\left( \frac{Y_{t}}{Y}\right) ^{\kappa _{y}}\right] ^{1-\kappa _{i}}$
(Taylor rule)

\item $1+i_{t}=R_{t+1}\mathbb{E}_{t}(\Pi _{t+1})$ (Fischer equation)

\item $Y_{t}=C_{t}++G_{t}+I_{c,t}+I_{d,t}+\frac{\gamma _{i,c}}{2}\left( 
\frac{I_{net,c,t}+I_{c}}{I_{net,c,t-1}+I_{c}}-1\right) ^{2}\left(
I_{net,c,\tau }+I_{c}\right) +\frac{\gamma _{i,d}}{2}\left( \frac{%
I_{net,d,t}+I_{d}}{I_{net,d,t-1}+I_{d}}-1\right) ^{2}\left( I_{net,d,\tau
}+I_{d}\right) +\frac{\gamma _{p}}{2}\left( \frac{\Pi _{t}}{\Pi
_{t-1}^{\kappa _{p}}\Pi ^{1-\kappa _{p}}}-1\right) ^{2}Y_{t}$ (budget
constraint of the economy)

\item $G_{t}=\overset{\_}{G}$ (public consumption)...da cambiare quando
inseriamo credit policy
\end{enumerate}

\textbf{Number of variables: 43}

\begin{enumerate}
\item $C_{t}$ 44

\item $\lambda _{t}$ 2

\item $L_{t}$ 16

\item $W_{t}$ \ 3

\item $\Lambda _{t,t+1}$ 5

\item $R_{t}$ \ 4

\item $U_{t}$ 6

\item $Y_{t}$ 7

\item $Y_{c,t}$ 12

\item $Y_{d,t}$ 13

\item $\Pi _{t}$ 42

\item $MC_{t}$ 8

\item $p_{c,t}$ 9

\item $p_{d,t}$ 10

\item $K_{c,t}$ 21

\item $K_{d,t}$ 22

\item $L_{c,t}$ 14

\item $L_{d,t}$ 15

\item $U_{c,t}$ 17

\item $U_{d,t}$ 18

\item $R_{c,t}$ 19

\item $R_{d,t}$ 20

\item $Q_{c,t}$ 27

\item $Q_{d,t}$ 28

\item $S_{c,t}$ 34

\item $S_{d,t}$ 39

\item $I_{net,c,t}$ 23

\item $I_{net,d,t}$ 24

\item $I_{c,t}$ 25

\item $I_{d,t}$ 26

\item $\Xi _{t}$ 33

\item $B_{t}$ 29

\item $N_{t}$ 30

\item $N_{e,t}$ 31

\item $N_{n,t}$ 32

\item $\phi _{t}$ 35

\item $\Omega _{t+1}$ 36

\item $\nu _{t}$ 37

\item $\eta _{t}$ 38

\item $V_{t}$ 40

\item $G_{t}$ 46

\item $T_{t}$ 41

\item $i_{t}$ \ 43
\end{enumerate}

We have 9 additional equations and 9 additional variables related to the
stochastic part of the model:

\begin{enumerate}
\item $A_{c,t}=A_{c}e^{u_{a,c,{t}}+u_{a,t}}$ \ \ \ \ \ \ \ \ \ \ \ 

\item $A_{d,t}=A_{c}e^{u_{a,d,{t}}+u_{a,t}}$ \ \ \ \ \ \ 

\item $\xi _{c,t\text{ }}=\xi _{c\text{ }}e^{u_{\xi ,c,{t}}}$

\item $\xi _{d,t\text{ }}=\xi _{d\text{ }}e^{u_{\xi ,d,{t}}}$

\item $u_{a,t}=\rho _{a}u_{a,t-1}+\varepsilon _{a,t}$\ \ \ \ \ \ $\ \ \ \ \
\ \ \ \ \ \ \ \ \ \ \ \ \ \ \ \ \ \ \ \ \ $

\item $u_{a,c,t}=\rho _{a,c}u_{a,c,t-1}+\varepsilon _{a,c,t}$

\item $u_{a,d,t}=\rho _{a,d}u_{a,d,t-1}+\varepsilon _{a,d,t}$

\item $u_{\xi ,c,t}=\rho _{\xi ,c}u_{\xi ,c,t-1}+\varepsilon _{\xi ,c,t}$

\item $u_{\xi ,d,t}=\rho _{\xi ,d}u_{\xi ,d,t-1}+\varepsilon _{\xi ,d,t}$
\end{enumerate}

$\bigskip $

$\tau _{W,t}$ $\tau _{c,t}$ $\tau _{d,t}$ (se fisse vanno incluse nella
tavola dei parametri)

Scale parameters: $A_{c}$, $A_{d}$, $\xi _{c}$, $\xi _{d}$, $\chi _{l}$. We
need 5 restrictions.

\subsection{Calibration and Steady state}

\begin{center}
\begin{tabular}{lll}
Parameter & Value & Description \\ 
$\alpha _{c}$ &  & capital share in the clean sector \\ 
$\alpha _{d}$ &  & capital share in the dirty sector \\ 
$\beta $ & 0.99 & discount rate \\ 
$\gamma _{i,c}$ &  & inverse elasticity of net investment to the price of
clean capital \\ 
$\gamma _{i,d}$ &  & inverse elasticity of net investment to the price of
dirty capital \\ 
$\gamma _{p}$ &  & degree of price rigidities \\ 
$\delta _{0}$ &  &  \\ 
$\delta _{1}$ &  &  \\ 
$\delta _{2}$ &  &  \\ 
$\varepsilon $ &  & elasticity of substitution between clean and dirty inputs
\\ 
$\theta $ & 0.972 & survival rate of the bankers \\ 
$\kappa _{i}$ & 0.8 & smoothing parameter of theTaylor rule \\ 
$\kappa _{p}$ & 1 & price backward indexation \\ 
$\kappa _{y}$ & 0.5 & output gap coefficient of theTaylor rule \\ 
$\kappa _{\pi }$ & 1.5 & inflation coefficient of the Taylor rule \\ 
$\varphi _{l}$ & 0.276 & inverse of the Frisch elasticity of labor supply \\ 
$\chi _{l}$ & SCALA & disutility of labor \\ 
$\rho $ & 0.381 & fraction of assets that can be diverted \\ 
$\sigma $ &  & elasticity of substitution between final goods varieties \\ 
$\omega $ & 0.002 & starting up transfer \\ 
$A_{c}$ & SCALA & steady state TFP in the clean sector \\ 
$A_{d}$ & SCALA & steady state TFP in the dirty sector \\ 
$ef_{c}$ &  & efficiency of clean inputs \\ 
$ef_{d}$ &  & efficiency of dirty inputs \\ 
$g$ & 0.2 & government share in GDP \\ 
$h_{c}$ &  & habit parameter \\ 
$i$ &  & steady-state nominal interest rate \\ 
$I_{c}$ &  & steady-state investment in the clean sector \\ 
$I_{d}$ &  & steady-state investment in the dirty sector \\ 
$s_{d}$ &  & proportion of dirty inputs in the portfolio \\ 
$U_{c}$ & 1 & steady-state utilization rate of capital in the clean sector
\\ 
$U_{d}$ & 1 & steady-state utilization rate of capital in the dirty sector
\\ 
$Y$ &  & steady-state output \\ 
$\xi _{c}$ & SCALA & steady-state quality of capital in the clean sector \\ 
$\xi _{d}$ & SCALA & steady-state quality of capital in the dirty sector \\ 
$\Pi $ & 1 & steady-state inflation \\ 
$\rho _{a}$ & 0.95 & persistence of productivity shock \\ 
$\rho _{a,c}$ &  & persistence of productivity shock in the clean sector \\ 
$\rho _{a,d}$ &  & persistence of productivity shock in the dirty sector \\ 
$\rho _{\xi ,c}$ & 0.66 & persistence of capital quality shock in the clean
sector \\ 
$\rho _{\xi ,d}$ & 0.66 & persistence of capital quality shock in the dirty
sector \\ 
$\sigma _{a}$ &  & standard deviation of productvity shock \\ 
$\sigma _{a,c}$ &  & standard deviation of productvity shock in the clean
sector \\ 
$\sigma _{a,d}$ &  & standard deviation of productvity shock in the dirty
sector \\ 
$\sigma _{\xi ,c}$ & 0.05 & standard deviation of capital quality shock in
the clean sector \\ 
$\sigma _{\xi ,c}$ & 0.05 & standard deviation of capital quality shock in
the dirty sector%
\end{tabular}
\end{center}

\subsubsection{Steady-state model}

\begin{enumerate}
\item $\frac{1}{C-h_{c}\overset{\_}{C}}=\lambda $ \ (marginal utility of
consumption)

\item $\chi _{l}L^{\varphi _{l}}=\lambda W(1-\tau _{W})$ \ (labor supply)

\item $R=\frac{1}{\beta }$ \ (Euler's equation)

\item $\Lambda =1$ \ (discount factor)

\item $U=\ln \left( C-h_{c}\overset{\_}{C}\right) -\chi _{l}\frac{%
L^{1+\varphi _{l}}}{1+\varphi _{l}}+\beta U$ \ (welfare measure)

\item $Y=\left[ \rho ^{1/\varepsilon }\left( ef_{c}Y_{c}\right) ^{\left(
\varepsilon -1\right) /\varepsilon }+(1-\rho )^{1/\varepsilon }\left(
ef_{d}Y_{d}\right) ^{\left( \varepsilon -1\right) /\varepsilon }\right]
^{\varepsilon /(\varepsilon -1)}$\ \ (production function)

\item $MC=\frac{\sigma -1}{\sigma }$\ \ (NKPC)

\item $p_{c}\left( 1+\tau _{c}\right) =MC\rho ^{1/\varepsilon }ef_{c}\left(
ef_{c}Y_{c}\right) ^{\frac{\varepsilon -1}{\varepsilon }-1}Y^{-1}$ \ (demand
for clean input)

\item $p_{d}\left( 1+\tau _{d}\right) =MC\rho ^{1/\varepsilon }ef_{d}\left(
ef_{d}Y_{d}\right) ^{\frac{\varepsilon -1}{\varepsilon }-1}Y^{-1}$ \ (demand
for dirty input)

\item $Y_{c}=A_{c}\left( U_{c}\xi _{c}K_{c}\right) ^{\alpha
_{c}}L_{c}^{1-\alpha _{c}}$ (production function clean sector)

\item $Y_{d}=A_{d}\left( U_{d}\xi _{d}K_{d}\right) ^{\alpha
_{d}}L_{d}^{1-\alpha _{d}}$ (production function dirty sector)

\item $W=p_{c}\left( 1-\alpha _{c}\right) \frac{Y_{c}}{L_{c}}$ (labor demand
clean sector)

\item $W=p_{d}\left( 1-\alpha _{d}\right) \frac{Y_{d}}{L_{d}}$ (labor demand
dirty sector)

\item $L=L_{d}+L_{c}$ (equilibrium condition in the labor market)

\item $\delta ^{\prime }(U_{c})\xi _{c}K_{c}=p_{c}\alpha _{c}\frac{Y_{c}}{%
U_{c}}$ (optimal utilization rate of capital clean sector) $\ \ \ \ \ \ \ \
\delta _{1}U_{c}^{\delta _{2}}\xi _{c}K_{c}=p_{c}\alpha _{c}\frac{Y_{c}}{%
U_{c}}$

\item $\delta ^{\prime }(U_{d})\xi _{d}K_{d}=p_{d}\alpha _{d}\frac{Y_{d}}{%
U_{d}}$ (optimal utilization rate of capital dirty sector) \ \ \ \ \ \ \ \ $%
\ \delta _{1}U_{d}^{\delta _{2}}\xi _{d}K_{d}=p_{d}\alpha _{d}\frac{Y_{d}}{%
U_{d}}$

\item $R_{c}=\frac{\left[ p_{c}\alpha _{c}\frac{Y_{c}}{\xi _{c}K_{c}}+\left(
Q_{c}-\delta (U_{c})\right) \right] \xi _{c}}{Q_{c}}$ (asset price equation
clean sector) $\ \ \ \ \ \ \ \ \ \ \ \ \ \ $Remember: $\delta _{0}+\frac{%
\delta _{1}}{1+\delta _{2}}U_{x,t}^{1+\delta _{2}}$

\item $R_{d}=\frac{\left[ p_{d}\alpha _{d}\frac{Y_{d}}{\xi _{d}K_{d}}+\left(
Q_{d}-\delta (U_{d})\right) \right] \xi _{d}}{Q_{d}}$ (asset price equation
dirty sector)

\item $K_{c}=S_{c}$ (arbitrage condition clean sector)

\item $K_{d}=S_{d}$ (arbitrage condition dirty sector)

\item $(1-\xi _{c})K_{c}=I_{net,c}$ (evolution of capital in the clean
sector)

\item $(1-\xi _{d})K_{d}=I_{net,d}$ (evolution of capital in the dirty
sector)

\item $I_{net,c}=I_{c}-\delta (U_{c})\xi _{c}K_{c}$ (evolution of net
investment in the clean sector) $\ \ \ \ \ \ $Remember to substitute: $%
\delta (U_{c})=\delta _{0}+\frac{\delta _{1}}{1+\delta _{2}}U_{c}^{1+\delta
_{2}}$

\item $I_{net,d}=I_{d}-\delta (U_{d})\xi _{d}K_{d}$ (evolution of net
investment in the dirty sector) \ \ \ \ \ \ Remember to substitute: $\delta
(U_{d})=\delta _{0}+\frac{\delta _{1}}{1+\delta _{2}}U_{d}^{1+\delta _{2}}$

\item $Q_{c}=1$ (price of unit of capital in the clean sector)

\item $Q_{d}=1$ (price of unit of capital in the dirty sector)

\item $\Xi =N+B$ (balance sheet)

\item $N=N_{e}+N_{n}$ (total net worth)

\item $N_{n}=\omega (Q_{c}S_{c}+Q_{d}S_{d})$ (net worth of new bankers)

\item $N_{e}=\theta \left\{ \left[ (R_{c}-R_{t})-\frac{Q_{d}S_{d}}{\Xi }%
(R_{c}-R_{d})\right] \phi _{t}+R_{t}\right\} N_{t}$ (net worth of existing
bankers)

\item $\Xi =Q_{c}S_{c}+Q_{d}S_{d}$ (total value of loans)

\item $Q_{c}S_{c}+Q_{d}S_{d}=\phi N$ (incentive constraint)

\item $\phi =\frac{\eta }{\rho -\nu }$ (private leverage ratio)

\item $\Omega =\beta \left[ 1+\theta (\nu \phi +\eta -1)\right] $ (effective
discount factor of bankers)

\item $\Omega (R_{c}-R)=\nu $

\item $\Omega R=\eta $

\item $\Omega (R_{c}-R_{d})=0\rightarrow s_{d}=Q_{d}S_{d}/\Xi $ \ \ \ 

\item $V=\nu \Xi +\eta N$ (bank objective function)

\item $G=T+\tau _{W}WL+\tau _{d}p_{d}Y_{d}-\tau _{c}p_{c}Y_{c}$ (government
budget constraint)

\item $1+i=(1+i)^{\kappa _{i}}\left[ (1+i)+1\right] ^{1-\kappa _{i}}$
(Taylor rule)

\item $1+i=R$ (Fischer equation)

\item $Y=C+G+I_{c}+I_{d}$ (budget constraint of the economy)

\item $G=\overset{\_}{G}$ (public consumption)...da cambiare quando
inseriamo credit policy
\end{enumerate}

\end{document}
